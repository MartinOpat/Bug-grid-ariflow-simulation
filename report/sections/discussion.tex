\section{Discussion} \label{section: discussion}


% TODO: Discuss newly added figure into the results
In this section, we discuss the results described in Section \ref{section: results}, their implications and causes in terms of the limitations of the implemented simulation. We also discuss the choices made in implementing the simulation model, as well as possible improvements. \\

Using the plots in Figure~\ref{fig: velocity-plot-holecount-7-holewidth-0.3} and Figure \ref{fig: density-holecount-7-holewidth-0.3}, we can qualitatively verify that the fluid simulation is running as expected. More of these plots for other sets of parameters can be found in \ref{appendix: A}. \\

To analyze the results quantitatively as well, two physical quantities were derived, implemented, and tracked over the duration of the simulation: the airflow through the window $Q$ and the total mass $m$ that flew through the window in a given time. \\
The airflow $Q$ is plotted against time in Figures \ref{fig: bound_plos_flowrate1}, \ref{fig: bound_plos_flowrate2}, and \ref{fig: bound_plos_flowrate3} on the right, with the corresponding bug grid boundary on the left. We can see an initial linear increase in airflow in all three cases. The air in the simulation starts at rest, so this initial linear trend is justified since the flow speed keeps increasing until reaching an equilibrium with the source term. An interesting observation is the appearance of damped oscillations in the transition between the initial (linearly increasing) flow and the final (equilibrium) flow. The interplay between the momentum of the fluid and the pressure gradients can explain these oscillations, and the fluid's viscosity explains the damping. \\
The total mass $m$ is displayed in Figure \ref{fig: flow_rate} for all tested parameter values.
% The plot in Figure \ref{fig: flow_rate}
This plot displays the relation between the airflow through the window and the grid parameters. One can immediately observe a severe reduction in the airflow caused by the presence of the grid. It is, however, difficult to read from this plot whether this reduction is merely proportional to the area of the window that is blocked by the grid. 
This is why the plot in Figure \ref{fig: flow_rate_norm} showing the normalized airflow divides the ``absolute" airflow measured through the window by the percentage of the window that is open, i.e., not blocked by the grid. This normalization allows for a more direct comparison between the different grid sizes, but most importantly, it allows for a comparison between the window with and without a grid. If the grid were only to block the amount of airflow proportional to the window area it covers, the normalized airflow would be constant for all grid sizes. However, looking at Figure \ref{fig: flow_rate_norm}, we can see that even the normalized airflow is significantly decreased by the presence of the grid. This result implies the presence of the grid introduces additional resistance and turbulence to the airflow.\\


% TODO: boundary conditions were a struggle, cite that paper we used to solve that

% TODO: Improvement, write specific dedicated solver specifically for this case and/or use packages specificallt for fluids simulation, not ust pde solving

While the obtained results discussed above nicely show the effect of the grid on the airflow, there are several limitations to the simulation model that need to be addressed. The most significant limitation, and the motivation behind many of the choices made in the implementation, is the computational complexity of the simulation, combined with a lack of computational resources. The most drastic decision made to mitigate this limitation was to switch from a 2D to a 3D simulation. A 2D simulation is significantly less computationally expensive, however, with a 2D simulation one is taking the risk of possibly missing out on important 3D effects. If it were these 3D effects that were the main cause of the reduction of the normalized airflow, then the results obtained from the 2D simulation would not be significant. Luckily, as we discussed above, an unpropotional reduction in airflow was also observed in the 2D simulation, implying that any solely 3D effects that were missed out are negligible w.r.t. the qualitative goals of this study. Of course, it is worth to note that the 3D effects would still be significant in any quantitative measurements.  \\

Another issue experienced during the implementation was the numerical instability of the simulation. The used time integration scheme was a simple forward Euler scheme, which is known to be unstable for the Navier-Stokes equations, as was already discussed in Section \ref{section: methods}. One implication of these instabilities was already mentioned in Section \ref{section: implementation} - chequerboard oscillations. These oscillations are a result of the discretization of the fields on the grid, and are a common issue in fluid simulations cite{harlow1965numerical}. A computationally cheap way to mitigate these oscillations is to use artificial viscosity, which was implemented in the simulation. A better, but more computationally expensive, way to mitigate these oscillations would be to use a higher-order time integration scheme, such as the Runge-Kutta 4 scheme \cite{RungeKutta}. Another observed issue related to the numerical instability was the presence of high frequency noise and large gradients in the fields. Some simpe signal processing techniques were attempted to smooth out the fields, but the results were not satisfactory. Instead, the combination of the artificial viscosity and increased grid resolution that became possible due to the switch to 2D was used to resolve this issue. \\

One might pose the question, if it is the Navier-Stokes equations that are causing the numerical instability, why not use a different set of equations, such as the Euler equations? The Euler equations are a simplified version of the Navier-Stokes equations, where the viscosity term is neglected \cite{eulerEqs}. This simplification could indeed make the simulation more stable, but it would also make it less accurate. The viscosity term in the Navier-Stokes equations is crucial for the simulation of the airflow through the window as the viscosity is responsible for the reduction in airflow caused by the grid, since it is the viscosity that causes the resistance and turbulence in the flow. Another choice worth discussing is the decision to simulate the flow as a compressible fluid. An argument could be made that at low speed, such as the ones usually experienced in everyday life, air can be for the most part considered incompressible. This would simplify the simulation, as the density would be constant and the simulation would be less computationally expensive. However, this change would not necessarily make the simulation more numerically stable or accurate, and it is even debatable whether this approach would less computationally expensive, as in many cases a Poisson equation for the pressure would need to be solved cite{incompressible}. For those reasons, the decision was made to implement a compressible fluid simulation, as described in Sections \ref{section: methods} and \ref{section: implementation}. \\

Additionally, the correct choice of boundary conditions is crucial for the stability and accuracy of the simulation. Here by correct, such boundary conditions are meant that they are computationally stable while still being physically meaningful. 