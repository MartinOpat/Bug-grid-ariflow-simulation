\subsection*{Implemented visualization techniques} \label{subsection: implemented visualization techniques}
% \addcontentsline{toc}{subsection}{Implemented visualization techniques}

% The different vector field visualization techniques implemented for the fluid simulation \gls{Android} application are presented in this appendix. Figure \ref{fig:simple-visualization} shows the simple vector field visualization technique that is used to measure the performance (see section \ref{subsection: vector field rendering}). In contrast, Figure \ref{fig:box-visualization} presents a visualization technique where the outline of the vector field is displayed as an opaque cuboid. In both methods, the colors are based on the vectors' directions and mapped using an HSV mapping.

% \begin{figure}[H]
%     \centering
%     \begin{subfigure}[b]{0.23\textwidth}
%       \includegraphics[width=\textwidth]{figures/screenshots/appendix/dg_2d.jpg}
%       \caption{Double gyre field x-y plane view.}
%     \end{subfigure}
%     \hspace{1cm}
%     \begin{subfigure}[b]{0.23\textwidth}
%       \includegraphics[width=\textwidth]{figures/screenshots/appendix/dg_3d.jpg}
%       \caption{Double gyre field rotated.}
%     \end{subfigure}
%     \hspace{1cm}
%     \begin{subfigure}[b]{0.23\textwidth}
%         \includegraphics[width=\textwidth]{figures/screenshots/appendix/dg_side.jpg}
%         \caption{Double gyre field z-y plane view.}
%       \end{subfigure}
%       \begin{subfigure}[b]{0.23\textwidth}
%         \includegraphics[width=\textwidth]{figures/screenshots/appendix/perlin_2d.jpg}
%         \caption{Perlin noise field x-y plane view.}
%       \end{subfigure}
%       \hspace{1cm}
%       \begin{subfigure}[b]{0.23\textwidth}
%         \includegraphics[width=\textwidth]{figures/screenshots/appendix/perlin_3d.jpg}
%         \caption{Perlin noise field rotated.}
%       \end{subfigure}
%       \hspace{1cm}
%       \begin{subfigure}[b]{0.23\textwidth}
%           \includegraphics[width=\textwidth]{figures/screenshots/appendix/perlin_side.jpg}
%           \caption{Perlin noise field z-y plane view.}
%         \end{subfigure}
%     \caption{Simple vector field visualization technique where vectors are rendered as straight lines colored based on the vector's direction.}
%     \label{fig:simple-visualization}
% \end{figure}

% The visualization method depicted in Figure \ref{fig:box-visualization} performs a 3D texture interpolation on the graphical fragments to render the opaque cuboid. The cuboid can be sliced using clipping planes orthogonal to the $x$ and $y$ axes. The position of the clipping planes can be adjusted using the sliders on the bottom and right sides of the screen. 
% \begin{figure}[H]
%     \centering
%     \begin{subfigure}[b]{0.23\textwidth}
%       \includegraphics[width=\textwidth]{figures/screenshots/appendix/dg_2d_box_cut_alt.jpg}
%       \caption{Double gyre field x-y plane view.}
%     \end{subfigure}
%     \hspace{1cm}
%     \begin{subfigure}[b]{0.23\textwidth}
%       \includegraphics[width=\textwidth]{figures/screenshots/appendix/dg_3d_box_cut_alt.jpg}
%       \caption{Double gyre field rotated.}
%     \end{subfigure}
%     \hspace{1cm}
%     \begin{subfigure}[b]{0.23\textwidth}
%         \includegraphics[width=\textwidth]{figures/screenshots/appendix/dg_side_box_cut.jpg}
%         \caption{Double gyre field z-y plane view.}
%       \end{subfigure}
%       \begin{subfigure}[b]{0.23\textwidth}
%         \includegraphics[width=\textwidth]{figures/screenshots/appendix/perlin_2d_box.jpg}
%         \caption{Perlin noise field x-y plane view.}
%       \end{subfigure}
%       \hspace{1cm}
%       \begin{subfigure}[b]{0.23\textwidth}
%         \includegraphics[width=\textwidth]{figures/screenshots/appendix/perlin_3d_box_cut.jpg}
%         \caption{Perlin noise field rotated.}
%       \end{subfigure}
%       \hspace{1cm}
%       \begin{subfigure}[b]{0.23\textwidth}
%           \includegraphics[width=\textwidth]{figures/screenshots/appendix/perlin_side_box.jpg}
%           \caption{Perlin noise field z-y plane view.}
%         \end{subfigure}
%     \caption{3D Texture slicing visualization technique.}
%     \label{fig:box-visualization}
% \end{figure}

% Additionally, line integral convolution visualization is shown in Figure \ref{fig:lic-visualization}. This method integrates a grey-scale Perlin noise texture over the vector field's Eulerian vertices, resulting in the rendered images.

% \begin{figure}[H]
%     \centering
%     \begin{subfigure}[b]{0.23\textwidth}
%       \includegraphics[width=\textwidth]{figures/screenshots/appendix/dg_2d_lic.jpg}
%       \caption{Double gyre field x-y plane view.}
%     \end{subfigure}
%     \hspace{1cm}
%     \begin{subfigure}[b]{0.23\textwidth}
%       \includegraphics[width=\textwidth]{figures/screenshots/appendix/dg_3d_lic.jpg}
%       \caption{Double gyre field rotated.}
%     \end{subfigure}
%     \hspace{1cm}
%     \begin{subfigure}[b]{0.23\textwidth}
%         \includegraphics[width=\textwidth]{figures/screenshots/appendix/dg_side_lic.jpg}
%         \caption{Double gyre field \mbox{oblique z-y plane view.}}
%       \end{subfigure}
%       \begin{subfigure}[b]{0.23\textwidth}
%         \includegraphics[width=\textwidth]{figures/screenshots/appendix/perlin_2d_lic.jpg}
%         \caption{Perlin noise field x-y plane view.}
%       \end{subfigure}
%       \hspace{1cm}
%       \begin{subfigure}[b]{0.23\textwidth}
%         \includegraphics[width=\textwidth]{figures/screenshots/appendix/perlin_3d_lic.jpg}
%         \caption{Perlin noise field rotated.}
%       \end{subfigure}
%       \hspace{1cm}
%       \begin{subfigure}[b]{0.23\textwidth}
%           \includegraphics[width=\textwidth]{figures/screenshots/appendix/perlin_side_lic.jpg}
%           \caption{Perlin noise field \mbox{oblique z-y plane view.}}
%         \end{subfigure}
%     \caption{Line integral convolution visualization technique.}
%     \label{fig:lic-visualization}
% \end{figure}