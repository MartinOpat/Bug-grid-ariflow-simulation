\section{Conclusion} \label{section: conclusion}
In this report we described the mathematical underpinnings of a numerical model for compressible fluid flow simulation. The physical constraints, initial and boundary conditions were explained. The model was implemented in Python using the finite difference method for space discretization, and forward Euler method for time discretization. The Py-PDE Python library was used to perform the numerical integration, and Just-In-Time compilation into C was performed to optimize the implementation.\\

The simulation was run for several parameters of the bug grid, changing in the number of holes and the sizes of the holes. Subsequently, the data was processed, visualized, analyzed and discussed. The results showed that the flow of the fluid was affected by the presence of the bug grid. The flow was faster in the regions with larger holes, and slower in the regions with smaller holes. Additionally, the results showed that the flow was decreased unproportionately to the amount of the window that the bug grid covered. This result implies that the bug grid introduces additional resistance to the flow of the fluid. The reduction in airflow was between $50\%$ and $96\%$ for the tested parameters. \\

The limitations of the simulation model were discussed, and several possible improvements were proposed. Most notably, the numerical instability of the simulation was addressed by using artificial viscosity, and also by simulating the fluid flow in only two dimensions, which allowed for a higher grid resolution. More computational resources would allow for a more sophisticated integration method, such as the Runge-Kutta 4 scheme instead of the used forward Euler method.
Additionally, the simulation model could be further improved by using a more sophisticated partial differential equation solver.